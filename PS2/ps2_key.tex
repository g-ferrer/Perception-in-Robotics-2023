\documentclass[11pt, oneside, letterpaper]{article}
\usepackage{amsmath}
\usepackage{amssymb}
\usepackage{caption}
\usepackage{enumerate}
\usepackage{float}
\usepackage{gensymb}
\usepackage{graphicx}
\usepackage{subfig}
\usepackage[cm]{fullpage}
\RequirePackage[T1]{fontenc}
\RequirePackage{times}
\RequirePackage{charter}
\newcommand{\tasksep}{\noindent\underline{\hspace{\textwidth}}}

\begin{document}

\title{\huge\textbf{EECS568 Mobile Robotics: Methods \& Algorithms (W2017) PS2 Solutions}}
\author{Gonzalo Ferrer\\University of Michigan}
\date{\today}
\maketitle

\section*{Task 1: Landmark Localization (100 points)}

\subsection*{Part A}

The following are equations of $Q$, $M_t$, $G_t$, $V_t$, and $H_t$, respectively.

\begin{equation*}
\label{eqn:TimeStepT}
\begin{aligned}
Q &= \left(20 \times \frac{\pi}{180^{\circ}}\right)^2 \approx 0.1218 \\
M_t &= \begin{bmatrix}
\alpha_1 \delta_{rot1}^{2} + \alpha_2 \delta_{tran}^{2} & 0 & 0 \\
0 & \alpha_3 \delta_{tran}^{2} + \alpha_4 (\delta_{rot1}^{2} + \delta_{rot2}^{2}) & 0 \\
0 & 0 & \alpha_1 \delta_{rot1}^{2} + \alpha_2 \delta_{tran}^{2}
\end{bmatrix} \\
G_t &= \begin{bmatrix}
1 & 0 & -\delta_{tran}sin(\theta + \delta_{rot1}) \\
0 & 1 & \ \ \, \delta_{tran}cos(\theta + \delta_{rot1}) \\
0 & 0 & 1
\end{bmatrix} \\
V_t &= \begin{bmatrix}
-\delta_{tran}sin(\theta + \delta_{rot1}) & cos(\theta + \delta_{rot1}) & 0  \\
\ \ \, \delta_{tran}cos(\theta + \delta_{rot1}) & sin(\theta + \delta_{rot1}) & 0 \\
1 & 0 & 1
\end{bmatrix} \\
q &= (m_x - x)^2 + (m_y - y)^2 \\
H_t &= \begin{bmatrix}
\frac{(m_y - y)}{q} & \frac{-(m_x - x)}{q} & -1
\end{bmatrix}
\end{aligned}
\end{equation*}
The following are equations of $M_2$, $G_2$, and $V_2$, given that $u_2 = [0, 10, 0]^\top$, and $\mu_1 = [180, 50, 0]^\top$.

\begin{equation*}
\label{eqn:TimeStepTwo}
\begin{aligned}
M_2 &= \begin{bmatrix}
\frac{1}{10,000} & 0 & 0 \\
0 & \frac{1}{4} & 0 \\
0 & 0 & \frac{1}{10,000} \\
\end{bmatrix} \\
G_2 &= \begin{bmatrix}
1 & 0 & 0 \\
0 & 1 & 10 \\
0 & 0 & 1
\end{bmatrix} \\
V_2 &= \begin{bmatrix}
0 & 1 & 0 \\
10 & 0 & 0 \\
1 & 0 & 1
\end{bmatrix}
\end{aligned}
\end{equation*}

\subsection*{Part B}

Refer to the attached videos which illustrate the EKF and PF on the provided data.

\subsection*{Part C}

The plots here correspond to the videos in Part B.

\begin{figure}[!htb]
\minipage{0.32\textwidth}
    \includegraphics[width=\linewidth]{./figures/key/ekf_x_errors}
    \caption*{$x$-errors}
\endminipage\hfill
\minipage{0.32\textwidth}
    \includegraphics[width=\linewidth]{./figures/key/ekf_y_errors}
    \caption*{$y$-errors}
\endminipage\hfill
\minipage{0.32\textwidth}%
    \includegraphics[width=\linewidth]{./figures/key/ekf_t_errors}
    \caption*{$\theta$-errors}
\endminipage
\caption{The $+3\sigma$ and $-3\sigma$ plots of the state variables for the EKF.}
\end{figure}

\begin{figure}[!htb]
\minipage{0.32\textwidth}
    \includegraphics[width=\linewidth]{./figures/key/pf_x_errors}
    \caption*{$x$-errors}
\endminipage\hfill
\minipage{0.32\textwidth}
    \includegraphics[width=\linewidth]{./figures/key/pf_y_errors}
    \caption*{$y$-errors}
\endminipage\hfill
\minipage{0.32\textwidth}%
    \includegraphics[width=\linewidth]{./figures/key/pf_t_errors}
    \caption*{$\theta$-errors}
\endminipage
\caption{The $+3\sigma$ and $-3\sigma$ plots of the state variables for the PF.}
\end{figure}

\subsection*{Part D}

\subsubsection*{What happens as sensor or motion noise go toward zero?}

Despite the randomness in Figure \ref{fig:NoiseApproachingZero}, the average localization error decreases as the motion and sensor noise parameters approach zero. Here, the average localization error is the mean of the $L_2$ norm between the estimate state and the ground truth.

This behavior can be explained as follows: (1) when the motion model's noise approaches zero, the filter tends to trust the odometry over the bearing measurements; (2) when the sensor noise approaches zero, the filter will tend to trust the measurements over the odometry.

\begin{figure}[!htb]
\minipage{0.5\textwidth}
    \includegraphics[width=\linewidth]{./figures/key/partD1}
    \caption*{Motion noise approaching zero.}
\endminipage\hfill
\minipage{0.5\textwidth}
    \includegraphics[width=\linewidth]{./figures/key/partD2}
    \caption*{Sensor noise approaching zero.}
\endminipage\hfill
\caption{Localization error as the noise parameters go to zero.}
\label{fig:NoiseApproachingZero}
\end{figure}

\subsubsection*{What happens as the number of particles decrease?}

Despite the randomness in Figure \ref{fig:nParticlesDecreasing}, the average localization error increases as the number of particles decrease.

\begin{figure}[!htb]
\centering
\includegraphics[scale=0.5]{./figures/key/partD3}
\caption{Localization error as the number of particles decrease.}
\label{fig:nParticlesDecreasing}
\end{figure}

\subsubsection*{What happens if the filter underestimates/overestimates the noise parameters?}

Figure \ref{fig:EstimatingNoise} illustrates underestimating and overestimating the motion noise parameters in the EKF. It's evident that the filter performs poorly in both cases.

\begin{figure}[!htb]
\minipage{0.5\textwidth}
    \includegraphics[width=\linewidth]{./figures/key/partD4}
    \caption*{Underestimating Noise}
\endminipage\hfill
\minipage{0.5\textwidth}
    \includegraphics[width=\linewidth]{./figures/key/partD5}
    \caption*{Overestimating Noise}
\endminipage\hfill
\caption{Underestimate/Overestimated motion noise parameters.}
\label{fig:EstimatingNoise}
\end{figure}


\subsection*{Part E}

The attached \texttt{gl.avi} video shows a demonstration of the global localization problem using the particle filter. The filter uses a thousand particles uniformly generated throughout the field and eventually converges to the location of the robot.

\end{document}
